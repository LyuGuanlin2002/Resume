%%
%% Copyright (c) 2018-2019 Weitian LI <wt@liwt.net>
%% CC BY 4.0 License
%%
%% Created: 2018-04-11
%%

% Chinese version
\documentclass[zh]{resume}
\graphicspath{{./figure/}{./figures/}{./image/}{./images/}{./graphics/}{./graphic/}{./pictures/}{./picture/}}%提供多种图片路径


% Adjust icon size (default: same size as the text)
\iconsize{\Large}

% File information shown at the footer of the last page
\fileinfo{%
  \color{basecolor}
  \faCopyright~2023--2024, Guanlin Lyu \hspace{0.5em}
  \creativecommons{by}{4.0} \hspace{0.5em}
  \faEdit~\today
}
\name{冠霖}{吕}

%\keywords{BSD, Linux, Programming, Python, C, Shell, DevOps, SysAdmin}

% \tagline{\icon{\faBinoculars}} <position-to-look-for>}
% \tagline{<current-position>}

\photo{purple.png}
\logoimage{oplogo.pdf}

\profile{
  \mobile{138-4054-9691}
  \email{1793930480@qq.com}\\
  \university{四川大学}
  \degree{工程力学}
  \birthday{2002-04-03}
  \home{辽宁 \textbullet 沈阳}
  % Custom information:
  % \icontext{<icon>}{<text>}
  % \iconlink{<icon>}{<link>}{<text>}
}


\hobby{
\leftitem{网球}
\leftitem{羽毛球}
\leftitem{绘画}
}

\certificate{
\leftitem{英语四级证书}
\leftitem{英语六级证书}
}

\saying{窝嫩叠。}




\begin{document}

\universitylogo

\begin{paracol}{2}

\makeheader


\switchcolumn

%======================================================================
% Summary & Objectives
%======================================================================
\Name
工程力学强基计划本科生,必修课平均成绩达到~87.904,必修绩点为~3.648,有扎实的力学、数学基础,多次参加全国力学竞赛并获奖;具有较好的语言基础,英语已达到六级水平;具有一定的计算机编程基础。

%======================================================================
\sectionTitle{技能和语言}{\faWrench}
%======================================================================
\begin{competences}
  \comptence{三维建模软件}{%
    SolidWorks, Creo
  }
  \comptence{数值模拟软件}{%
    ABAQUS, Comsol
  }
  \comptence{数学计算软件}{%
    MATLAB, Maple, Mathematica
  }
  \comptence{{\color{basecolor}\large\faKeyboard}~编程语言}{%
    Python, C, C++
  }
  \comptence{{\color{basecolor}\large\faLanguage}~外语水平}{%
    \textbf{英语} --- 读写(优良),听说(日常交流)
  }
\end{competences}

%======================================================================
\sectionTitle{教育背景}{\faGraduationCap}
%======================================================================
\begin{educations}
  \education%
    {至今}%
    [2020.09]%
    {四川大学}%
    {建筑与环境学院}%
    {工程力学(强基计划)}%
    {本科在读}
\end{educations}


%======================================================================
\sectionTitle{获奖荣誉}{\faBookOpen}
%======================================================================
\begin{itemize}
  \item 2021~年四川省大学生周培源力学竞赛二等奖
  \item 2022~年四川省第九届孙训方大学生力学竞赛本科及以上组三等奖
  \item 2023~年四川省大学生周培源力学竞赛一等奖
  \item 第十四届全国周培源大学生力学竞赛(个人赛)三等奖
  \item 第十四届全国周培源大学生力学竞赛(团体赛)优胜奖
  \item 第一届全国大学生“力学+X”创新实践研讨会创新作品
  \item 2020-2021~学年四川大学综合三等奖学金
  \item 2021-2022~学年四川大学综合二等奖学金
  \item 2020-2021~学年四川大学建筑与环境学院优秀学生
\end{itemize}

%======================================================================
\sectionTitle{科研经历}{\faCube}
%======================================================================
\begin{itemize}
  \item 基于数据驱动的燃料结构试验件冲击载荷预测与反演软件平台开发,参与
  \item 基于超声原位实验技术及机器学习的疲劳短裂纹扩展机理研究,参与
  \item 定向凝固~DZ125~合金的超高周疲劳性能及裂纹演化机理研究,负责
  \item 基于晶体相场法的微裂纹演化模拟,负责
\end{itemize}


%======================================================================
\sectionTitle{科研成果}{\faAtom}
%======================================================================
\begin{itemize}
  \item 发表~SCI~论文~A machine learning study on the fatigue crack path of short crack on an~$\alpha$~titanium alloy, 第二作者
  \item 发表~SCI~论文~Advances in dynamic load identification based on data-driven techniques, 第三作者
  \item 申请专利:一种热梯度超声疲劳试验系统(申请中)
\end{itemize}

%======================================================================
\sectionTitle{实习经历}{\faBriefcase}
%======================================================================
\begin{experiences}
  \experience%
    []%
    {2023.07}%
    {本科毕业实习}%
    []
\end{experiences}





\end{paracol}


\end{document}
